\documentclass[a4paper,10pt]{article}

\usepackage[margin=2cm]{geometry}
\usepackage{graphicx}
\usepackage{caption}
\usepackage{setspace}

\setlength{\parindent}{0pt}
\setlength{\parskip}{6pt}

\title{\textbf{Comparison of Gene Expression Profiles in ALL and AML}}
\author{Sofiya Aniskevich}
\date{23.01.2026}

\begin{document}
\maketitle

\section*{Introduction and Data}

Gene expression profiling plays a key role in modern bioinformatics and cancer research.
In this project, we analyze a classic leukemia dataset published by Golub et al. (1999),
which demonstrated that acute lymphoblastic leukemia (ALL) and acute myeloid leukemia (AML)
can be distinguished based on gene expression patterns measured using DNA microarrays.

The dataset contains expression values for several thousand genes measured in bone marrow
and peripheral blood samples. Mean expression values were calculated separately for ALL
and AML samples, resulting in two distributions representing global gene expression
profiles for each leukemia subtype.

\section*{Methods}

All analyses were performed in a Linux terminal environment.
Mean gene expression values for ALL and AML samples were computed using command-line
text processing tools.
Specifically, the \texttt{awk} utility was used to parse the expression matrix and
calculate per-gene mean expression levels for each leukemia subtype.

Data visualization was carried out using \texttt{gnuplot}.
The resulting mean expression values were summarized using boxplots, which provide
a robust statistical representation of the distribution of gene expression levels,
including medians, quartiles, and extreme values.

\section*{Results}

Figure~1 presents boxplots of mean gene expression values for ALL and AML.
Each point corresponds to a single gene, while the boxplots summarize the overall
distribution of expression levels.

\begin{center}
\includegraphics[width=0.75\linewidth]{leukemia_expression_boxplot.png}

\small
\textbf{Figure 1.} Distribution of mean gene expression values in ALL and AML samples.
\end{center}

The median expression levels are similar for both leukemia types, indicating that
most genes exhibit comparable expression across ALL and AML.
However, both distributions contain extreme values, corresponding to a subset of genes
with very high or very low expression levels.
Notably, AML shows slightly more pronounced high-expression outliers, suggesting the
presence of genes that may be specifically overexpressed in myeloid leukemia.

\section*{Conclusion}

This analysis demonstrates that while global gene expression patterns in ALL and AML
are broadly similar, biologically relevant differences arise from a limited number of
genes with extreme expression values. Such genes may serve as potential markers for
leukemia classification.

\end{document}
